\documentclass{ltjsarticle}

% 和文フォント設定
\usepackage[ms]{luatexja-preset}

% -- use following lines instead of above luatexja-preset.
% -- see http://sourceforge.jp/projects/luatex-ja/wiki/LuaTeX-ja%E3%81%AE%E4%BD%BF%E3%81%84%E6%96%B9
%\usepackage{luatexja-fontspec}
%\setmainjfont{Kozuka Mincho Pr6N} %<-- it has no unicode math char.
%\setsansjfont{MS Gothic}

% 欧文フォント設定
\usepackage{fontspec}
\setmainfont[Scale=MatchLowercase]{DejaVu Sans} % \rmfamily のフォント
\setsansfont[Scale=MatchLowercase]{DejaVu Sans}  % \sffamily のフォント
\setmonofont[Scale=MatchLowercase]{MS Gothic}       % \ttfamily のフォント

%\usepackage{fancyvrb}
% \DefineVerbatimEnvironment{code}{Verbatim}{}

\begin{document}

\title{圏論学習の記録}
\author{木下修司}
\date{2014年10月28日}
\maketitle

\section{はじめに}

\subsection{経緯}

圏論(Category Theory)の勉強を2014年6月から本格的に始めた\footnote{4月頃からやろうと話していたが、学振の申請が一段落してから始まった。ちなみに落ちた。}。
目標は曖昧なのだが、情報科学、特に関数型言語研究の世界で
利用されている概念をひととおり理解していろいろ使えるようになることである。
たまたま圏論を勉強したいという動機をNAISTの博士課程に在籍する宮原一喜さん\footnote{私のNAIST時代の研究室の1年先輩である}も
持っていたので、ふたりで本を読みつつ、月1、2回のペースでskypeするということになった。

ただし、圏論の道はなかなか険しい。一般に数学の本を読むというのは時間がかかる作業なのだが、
あまりにゆっくりで、かつ全貌が見通せない環境にいると人間は苦しむ。
吹雪の八甲田山のように。
我々もしかりで、6月当初はAwodeyを読み進めていたのだが、いつまでたってもNatural Transformationのような重要な概念が登場しないことに
苦しんだ。そこで、主要な概念がどんどん登場するMacLane本を読むことにしたのだが、これはこれで文中の例や練習問題がかなり難しい
という問題があることに気付いた。ここまで先週10月24日の話。

そこで、MacLane本の定義を読み進めて(時には書き進めて)理解しつつ、関連するAwodey本の部分や数学セミナー連載「圏論の歩き方」を
参照し、とりあえずは主要な概念を知る(理解してすらすら言える、というレベルをすぐには求めない)ことに重点を置くことにした。
2週間でMacLane本の1章分をとにかく読むということにした。
易きに流れた感はあるが、時には全体を先に見通すことも大事だと信じる。そして、適宜戻る。

\subsection{この文書とは}

私は私で、この文書を作成することにした。
基本的に時系列で、勉強した内容を書き記す場である。とりあえずは定義を参照するリファレンスとなることを
意図しているが、数学セミナー連載にも書いてあった「定義には書いていない気持ち」のようなものも書くことができれば
いいなと考えている。
また、TeXやらEmacsやらの練習も兼ねている。

圏論の勉強の仕方がよくわからずつまづいた人の、再学習の一助になれば幸いである。



\end{document}